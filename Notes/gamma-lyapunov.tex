\documentclass[11pt,reqno,a4paper]{amsart}
\usepackage{enumerate}
\usepackage{graphicx}
\usepackage[latin1]{inputenc}
\usepackage{amsmath,amssymb,MnSymbol,dsfont}
\usepackage{hyperref}
\usepackage{doi}
\usepackage[numbers,sort&compress]{natbib}
\usepackage{multicol}
\usepackage{tgtermes}
\usepackage{mathtools}
%\usepackage{amsthm}

\setlength{\textwidth}{17cm}
\setlength{\textheight}{23cm}
\setlength{\parindent}{0ex}
\setlength{\parskip}{1ex}

\setlength{\topmargin}{0mm}
\setlength{\headheight}{0mm}
\setlength{\headsep}{5mm}

\setlength{\marginparwidth}{0mm}
\setlength{\marginparpush}{0mm}
\setlength{\marginparsep}{0mm}
\setlength{\oddsidemargin}{0mm}
\setlength{\evensidemargin}{0mm}

%% only subsections
\makeatletter
\renewcommand\thesection{}
\renewcommand\thesubsection{\@arabic\c@subsection}
\makeatother

\newcommand{\eps}{\varepsilon}
\newcommand{\EE}{\mathbb{E}}
\newcommand{\RR}{\mathds{R}}
\newcommand{\NN}{\mathds{N}}
\newcommand{\tr}{\mathrm{tr}}

\newcommand{\argsup}{\operatornamewithlimits{argsup}}
\newcommand{\argmin}{\operatornamewithlimits{argmin}}

\newtheorem*{theorem}{Theorem}
\newtheorem{corollary}{Corollary}
\newtheorem{lemma}{Lemma}
\theoremstyle{definition}
\newtheorem*{remark}{Remark}

\begin{document}

\title{Operator $\Gamma$ for the quasi-linear approximation of QG}
\let\thefootnote\relax\footnotetext{Tobias Grafke, \textit{Date:}
  \today} % date only

\begin{abstract}
  Work out in a bit more detail the form of the linear operator in the
  quasi-linear approximation of the quasi-geostrophic equations in the
  $\beta$-plane.
\end{abstract}

\maketitle
\subsection{Setup}

Following \cite{bouchet-nardini-tangarife:2013}, the system of
equations we start with is
\begin{equation}
  \label{eq:QG-QL}
  \begin{cases}
    \partial_t U = \overline{v\omega}- U -\nu(-\partial_y^2)^p U\\
    d \omega = -\frac{1}{\alpha}\Gamma(U)\omega\,dt  + \sqrt{\frac{2}{\alpha}} \sigma dW
  \end{cases}
\end{equation}
where the second line is a stochastic PDE and an Ornstein-Uhlenbeck
process with drift 
\begin{equation}
  \label{eq:Gamma}
  \Gamma(U) = U\partial_x +(\partial_y^2 U-\beta)\partial_x \Delta^{-1} + \alpha + \nu(-\Delta)^p\,.
\end{equation}
We want to understand in detail what is happening with this equation
by drawing a parallel to a (finite-dimensional) Ornstein-Uhlenbeck
process. Consider, for $X\in\RR$,
\begin{equation}
  \label{eq:OU}
  dX = -\gamma X\,dt + \sigma\,dW\,,
\end{equation}
where $\gamma>0, \sigma>0$. Then, for long times, the variance of the
process (evaluated at the same times $t$), $\EE X^2 =
\sigma^2/(2\gamma)$, which can be derived for example by solving the
corresponding Fokker-Planck equation.

In higher dimensions, with $X\in\RR^n, \gamma\in\RR^{n\times n}$, and
$\sigma\in\RR^{n\times n}$, we must have that $\gamma$ is positive
definite, i.e.~$\forall\ x\ne0\in\RR^n, \langle x, \gamma x\rangle >
0$, or equivalently that all eigenvalues of $\gamma$ have positive
real part. Otherwise, the corresponding eigenmodes would be directions
into which the process would be unstable and therefore be driven
towards infinity. The variance of the process is then given by $\EE
X_i X_j = C_{ij}$, where the covariance matrix $C$ solves the
\emph{Lyapunov equation}
\begin{equation}
  \label{eq:Lyapunov}
  \gamma C + C \gamma^\dagger = 2 \chi\,,
\end{equation}
with $\chi = \sigma \sigma^\dagger$, and ${}^\dagger$ denotes the adjoint.

\subsection{Infinite-dimensional case}

Note that in the Ornstein-Uhlenbeck finite dimensional
case~(\ref{eq:OU}), the matrix-vector product is
\begin{equation}
  \label{eq:matrix-vector}
  \gamma X = \sum_i \gamma_{ij} X_j\,,
\end{equation}
where $\gamma$ depends on two indices $i$ and $j$. In the infinite
dimensional generalization of~(\ref{eq:QG-QL}), we replace indices by
coordinates, and sums with integrals. We have two spatial coordinates,
so our inner product becomes a double integral. We interpret
\begin{equation}
  \Gamma_U \omega = \iint \Gamma_U(x,y,x',y') \omega(x',y') dx' dy'
\end{equation}
and in the Lyapunov equation~(\ref{eq:Lyapunov}), the covariance
'matrix' becomes an object that depends on four coordinates,
$C(x,y,x',y')$. Of course, numerically, real space will be discretized
as well, and integrals become sums eventually, but we will deal with
that separately to not confuse the notation further.

We first note, for our specific $\Gamma_U$ in
equation~(\ref{eq:QG-QL}), the equation \emph{separates} in $k_x$. By
this we mean concretely: In a given Fourier-discretization in
$x$-direction, define a vector $(k_x)_i, i\in\{1,\dots,N_x\}$ that
defines the $i$-th discrete Fourier component, or $i$-th
\emph{mode}. The evolution equation for $\omega_{k_i}$ then depends
only on a right-hand-side with terms depending on the same mode $k_i$,
but no other modes $k_j, j\ne i$. Note that the same is not true in
$y$: For example, product terms like $U(y) \omega(x,y)$ become
convolutions in Fourier-space and therefore mix modes. But for $x$, we
can pass into Fourier-space and consider every mode $k_x$
separately. The evolution equation becomes
\begin{equation}
  d\omega_{k_x}(y) = \int \Gamma_{k_x}(y,y') \omega_{k_x}(y)\,dy'\ dt + \sqrt{\tfrac2\alpha}\sigma_{k_x} \,dW_{k_x}\,,
\end{equation}
(where the $U$ dependence of $\Gamma$ is supressed for
readability). Our problem is to find a proper representation of the
operator $\Gamma_k(y,y')$.

For this, note a few connections. First, we have, for a function
$f(x)$ and its Fourier coefficients $f_k$, that
\begin{equation}
  f(x) = \sum_k f_k e^{ikx} = \mathcal F[f_k](x)\,,\qquad f_k = \int f(x) e^{-ikx}\,dx = \mathcal F^{-1}[f(x)]_k
\end{equation}
where we define forward and backward Fourier transform operators as
$\mathcal F$ and $\mathcal F^{-1}$ and we are not too specific about
factors $2\pi$, domain size $L$, the precise values of the numerical
prefactors, the integral bounds and the summation bounds, and the
exact form needs to be worked out later.

Further, note that
\begin{equation}
  \label{eq:convolutions}
  \mathcal F^{-1} [f_k g_k](x) = \int f(z) g(x-z)\,dz\quad\text{and}\quad \mathcal F[f(x) g(x)]_k = \sum_{p,q} f_p g_{k-p}
\end{equation}
i.e.~products in Fourier-space become convolutions in Real-space, and
vice-versa.

We can now go ahead and try to evaluate the different terms of
$\Gamma_k$ as operator in this sense. Note that we want the
representation of $\Gamma$ in Real-space in $y$, because this is the
proper representation for Matlab's \texttt{lyap} (which uses
matrix-product convention~(\ref{eq:matrix-vector}), of course,
and not convolution, when interpreting the Lyapunov equation).

\subsubsection{Scalar multiplication}

Consider $\partial_t \omega = \gamma \omega$, for $\gamma \in
\RR$. Then of course
\begin{equation}
  \Gamma(y,y') = \gamma \delta(y-y')\,,
\end{equation}
because
\begin{equation}
  \int \Gamma(y,y') \omega(y')\,dy' = \gamma \int \delta(y-y')
  \omega(y')\,dy' = \gamma \omega(y)\,.
\end{equation}
In other words, scalars appear on the diagonal. The usage of the
$\delta$-function is a bit ugly, and maybe it should be written
discretely (since that is what we want in the end anyway). Note in
particular that $i k_x$ is a scalar in this sense, when computing
$x$-derivatives.

\subsubsection{Multiplication with function of $y$}

Consider we have $\partial_t \omega(y) = U(y) \omega(y)$. Then, with
the same argument as before,
\begin{equation}
  \Gamma(y,y') = U(y) \delta(y-y')\,.
\end{equation}
Thus, functions of $y$ appear on the diagonal, but the diagonal is
different for every entry.

\subsubsection{$y$-Derivatives}

A more complicated case is the derivative in $y$. Imagine $\partial_t
\omega = \partial_y \omega(y)$. We could use finite differences to
implement the derivative (i.e. central finite differences) and then
the implementation of $\Gamma$ as a matrix is straightforward. Maybe
we could implement this as a benchmark case. I'd generally prefer to
compute derivatives in Fourier-space, due to superior accuracy. For
this, define $D(y) = \mathcal F^{-1}[-ik_y](y)$ the inverse Fourier
transform of the derivative operator in Fourier space. Then we can use
the convolution connection~(\ref{eq:convolutions}, left) with $g_k =
-ik$ and $f_k = \omega_k$ to write
\begin{equation}
  \partial_y \omega(y) = \mathcal F^{-1}\left[\mathcal F[\partial_y \omega(y)]_{k_y}\right](y) = \mathcal F^{-1}\left[ -ik_y \omega_{k_y}\right](y) = \int \mathcal F^{-1}[-ik_y](y-y') \omega(y')\,dy' = \int D(y-y') \omega(y')\,dy'\,.
\end{equation}
We can therefore read off that
\begin{equation}
  \Gamma(y,y') = D(y-y') = \mathcal F^{-1}[-ik_y](y-y')\,.
\end{equation}
This $\Gamma$ is likely a full matrix. It must be the matrix that, if
multiplied by a vector of $y$ from the right, should yield its first
derivative. This test can be used in order to consider whether the
constants are chosen correctly, by taking a (discretized) function as
input for which the derivative is known analytically.

The other derivative operations that are actually needed can probably
be defined in the same way, i.e.~by using the inverse
Fourier-transform of the appropriate operator in Fourier-space (for
example $\beta (ik_x)/(k_x^2 + k_y^2)$). Similar to above, these parts
of the operator can be numerically tested by considering test cases
with analytical solutions.

\subsection{Usage}

Obviously, sums of terms lead to sums in $\Gamma$, so the above is
likely enough to build up all of $\Gamma$ from~(\ref{eq:Gamma}). This
can then be used in conjunction with Matlab's \texttt{lyap} to solve
the Lyapunov equation~(\ref{eq:Lyapunov}), and then solve the QG-QL
equation~(\ref{eq:QG-QL}) for infinite time separation.

Using this to compute the correlation
$\EE\left[\omega(y_1)\omega(y_2)\right]$ (equivalently, second
cumulant) instead of $\omega$ in equation~\eqref{eq:QG-QL}, I believe
we recover CE2 \cite{marston-conover-schneider:2008} or equivalently
S3T by \cite{farrell-ioannou:2003, farrell-ioannou:2007}. This also
opens the possibility to analyze emergence and stability of the jets
in some detail via linear stability analysis in the context of
atmosphere dynamics \cite{bakas-ioannou:2011, bakas-ioannou:2013} and
in the context of plasma physics \cite{parker-krommes:2013,
  parker-krommes:2014}.

\begin{multicols}{2}
  \footnotesize
  \setlength{\bibsep}{0.0pt}
  
  \bibliographystyle{apsrev4-1}
  \bibliography{bib}
\end{multicols}

\end{document}
