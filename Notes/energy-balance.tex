\documentclass[11pt]{amsart}
\usepackage{enumitem}
\usepackage{graphicx}
\usepackage[latin1]{inputenc}
\usepackage{amsmath}
\usepackage{amssymb}
\usepackage{MnSymbol}
\usepackage{hyperref}
\usepackage{doi}
\usepackage[numbers,sort&compress]{natbib}

\newcommand{\trace}{\mathrm{Tr}}

\pagestyle{plain}

\setlength{\textwidth}{15cm}
\setlength{\textheight}{24cm}
\setlength{\parindent}{0ex}
\setlength{\parskip}{1ex}

\setlength{\topmargin}{-5mm}
\setlength{\headheight}{0mm}
\setlength{\headsep}{0mm}

\setlength{\oddsidemargin}{1cm}
\setlength{\evensidemargin}{1cm}

\def\EE{\mathbb{E}}\def\PP{\mathbb{P}}
\def\RR{\mathbb{R}}

\begin{document}

\title{Energy Balance for QG-QL}
\date{\today}
\author{Tobias Grafke}

\begin{abstract}
  Energy Balance for the quasi-linear quasi-geostrophic equations, in
  particular to verify smallness assumptions for viscosity, and the
  scale separation between fluctuations and mean flow.
\end{abstract}

\maketitle

For the QG-QL system in properly rescaled variables is given by
\begin{subequations}
  \begin{align}
    \frac{\partial U}{\partial t} &= R_U - U - \frac{\nu}{\alpha} \Delta U\\
    d\omega &= -\frac1\alpha\left(U\partial_x \omega + (\partial_x \Delta^{-1}\omega)(\beta-U'') +\alpha \omega - \nu\Delta\omega\right)\,dt + \sqrt{\frac2\alpha}\sigma\,dW\,,
  \end{align}
\end{subequations}
where $R_U$ is the Reynolds stress,
\begin{equation}
  R_U = \frac1L\int v(x,y)\omega(x,y)\,dx
\end{equation}
for the domain $D=[0,L]^2$. Defining $C(x,y,x',y') =
\omega(x,y)\omega(x',y')$ we can obtain the Reynolds stress via
\begin{equation}
  R_U(y) = \int\left(\partial_x \Delta_y^{-1} C(x,y,x',y')\right)_{(x,y)=(x',y')}\,dx\,.
\end{equation}
The forcing normalization $\sigma$ is chosen such that $\EE[(\sigma
  dW_1)(-\Delta^{-1}\sigma dW_2)] = 1$.

Let $E_U$ be the energy in the zonal structures,
\begin{equation}
  E_U = \frac L2 \int \EE [U^2]\,dy\,.
\end{equation}
Then we obtain
\begin{equation}
  \frac{dE_U}{dt} = L\int \EE [UR_U]\,dy - 2E_U - \frac{L\nu}{\alpha} \int\EE[(\partial_y U)^2]\,dy
\end{equation}
and therefore
\begin{equation}
  E_U = \frac L2 \int \EE [UR_U]\,dy - \frac{L\nu}{2\alpha} \int\EE[(\partial_y U)^2]\,dy\,.
\end{equation}

Similarly, let $E_\omega$ be the energy of the turbulent fluctuations,
\begin{equation}
  E_\omega = \frac\alpha2\int \EE|\vec{v}|^2(x,y)\,dxdy = -\frac\alpha2\int\EE\left(\Delta_y^{-1} C(x,y,x',y')\right)_{(x,y)=(x',y')}\,dx\,dy\,.
\end{equation}
Then we obtain
\begin{align*}
  \frac{dE_\omega}{dt} &= -2E_\omega -\nu\int\EE[\omega^2]\,dx\,dy+\EE\underbrace{\int(U''-\beta)(\partial_x\Delta^{-1}\omega)(\Delta^{-1}\omega)\,dx\,dy}_{=0 \textrm{ (total deriv $\partial_x$)}}-\int\EE[U\omega (\partial_x\Delta^{-1}\omega)]\,dx\,dy+2\\
  &= -E_\omega -\frac\nu2\int\EE[C_{(x,y)=(x',y')}]\,dx\,dy-\tfrac12 \int\EE[U R_U]\,dx\,dy+1
\end{align*}
and therefore
\begin{equation}
  E_\omega= -\frac\nu2\int\EE[C_{(x,y)=(x',y')}]\,dx\,dy-\tfrac12 \int\EE[U R_U]\,dx\,dy+1
\end{equation}

Therefore, in total, for $E=E_U + E_\omega$,
\begin{equation}
  \frac{dE}{dt} = -E -\frac\nu2\int\EE[\omega^2]\,dx\,dy - \frac{L\nu}{2\alpha} \int\EE[(\partial_y U)^2]\,dy + 1
\end{equation}
so that in the stationary case we have
\begin{equation}
  E = 1 - \frac\nu2 \left(\int\EE[C]\,dx\,dy + \frac{L}{\alpha} \int\EE[(\partial_y U)^2]\,dy\right)
\end{equation}


\bibliographystyle{unsrtnat} 
\bibliography{bib}

\end{document}
