\documentclass[12pt]{amsart}
\usepackage[centertags]{amsmath}
\usepackage{amsfonts,amssymb,amsmath}
\usepackage{enumerate}
\usepackage[sort&compress,numbers]{natbib}
\usepackage{hyperref}
\usepackage{doi}
\usepackage{bbold}
\usepackage[latin1]{inputenc}

\usepackage{amsfonts,amsmath,amssymb,graphicx,color,amsthm}
\usepackage{tgtermes}

\textheight 8.5in
\textwidth 15cm
\oddsidemargin 0.75cm

\DeclareMathOperator{\sgn}{sgn}

\def\EE{\mathbb{E}}\def\PP{\mathbb{P}}

\begin{document}
\section{2D baratropic eqns on $\beta$-plane}

We consider 2D stochastically forced barotropic $\beta$-plane
turbulence,
\begin{equation}
  \label{eq:qg}
  \partial_t \omega + \vec{v}\cdot \nabla \omega + \beta v =
  -\lambda\omega - \nu (-\Delta)^p \omega + \sqrt{\gamma}\eta\,,
\end{equation}
 $\beta$-plane - a linear approximation of the coriolis parameter dependance on the latitude. (considering a linear topogrophy function $h(y) = \beta y$).Very useful approximation as it does not contribute non-linear terms to the system.
 
\subsection{Forcing}
 
The forcing was chosen to be spatially homogenous and isotropic which acts about a forcing wavenumber, $k_{*}$, and therefore only acts on a set on modes between $k_{a}$ and $k_{b}$ which are the limits of the spectrum. The spatial covariance, $\chi$, is defined as $\EE\left[\eta(\vec r_1, t_1)\eta(\vec
  r_2,t_2)\right]=\delta(t_2-t_1) \chi(|\vec r_2- \vec r_1|)$. The $\delta(t_2-t_1)$ term indicates rapid temporal decorrelation of the forcing. As the forcing is homogenous in $x$ and $y$, the spatial covariance is dependent on $x - x'$ and $y - y'$.
  
\subsection{De-coupling in $k_{x}$}

The vorticity can be expanded in zonal harmonics:

\begin{equation*}
  \omega(x,y,t) = \sum_{k=1}^{N_x} \hat\omega_k(y,t) e^{2\pi i k x/L}\,,
\end{equation*}
 
with $C_{k} = \EE[\hat\omega_k \hat\omega_k ^\dagger]$. Therefore, $\Gamma(U)$ can also be expanded in zonal harmonics. The forcing can also be expanded in zonal harmonics (chosen to be this way a priori).


\section{Computing $C_{k}$}

Computing $\Gamma(U)$:

\begin{equation*}
  \Gamma(U) = U\partial_x +(\partial_y^2 U-\beta)\partial_x \Delta^{-1} + \alpha + \nu(-\Delta)^p\,.
\end{equation*}

Taking x to be in Fourier space and y in Real space, computing for each mode $k_{x}$
\begin{align*}
 \Gamma(\mathbf{U})_k  = \mathrm{i} k \underbrace{\begin{pmatrix} U_1 & &\\&\ddots\\& & U_{N_x}\end{pmatrix}}_{\equiv\mathbb{U}}  & + \mathrm{i} k \underbrace{\begin{pmatrix} (\mathrm{d}^2U/\mathrm{d}y^2)_1 - \beta & &\\&\ddots\\& & (\mathrm{d}^2U/\mathrm{d}y^2)_{N_x} - \beta\end{pmatrix}}_{\equiv\mathbb{D_y^2U}-\beta}\big(\mathbb{D}_y^2-k^2\mathbb{I}\big)^{-1} \\
 &+ \alpha \mathbb{I} + \nu \big(-\mathbb{D}_y^2+k^2\mathbb{I}\big)^{p}
\end{align*}

where $\mathbb{D}_y^2$ is the matrix representation of the $\mathrm{d}^2/\mathrm{d}y^2$ operator.

We can then compute $C_{k}$ by solving the lyapunov eqn for each zonal mode:
\begin{equation*}
  \Gamma_k(U) C_k + C_k \Gamma_k(U)^\dagger = 2 \sigma_k \sigma_k^\dagger\
\end{equation*}

where $\sigma_k$ is a vector at the $k^{th}$ mode of the $\sigma$ matrix. \par
To compute the stationary energy, the full autocorrelation function $C(x,y,x'y')_{(x,y) = (x'y')}$ is required:
\begin{equation*}
E = 1 - \frac\nu2 \left(\int\EE[C]\,dx\,dy + \frac{L}{\alpha} \int\EE[(\partial_y U)^2]\,dy\right)
\end{equation*}
The $\int C \,dx$ term is equal to $C_0$ (k = 0 mode). Therefore the eqaution becomes:
\begin{equation*}
E = 1 - \frac\nu2 \left(\int\EE[C_0]\,dy + \frac{L}{\alpha} \int\EE[(\partial_y U)^2]\,dy\right).
\end{equation*}



\end{document}

